\documentclass[12pt,a4paper]{article}
\usepackage[utf8]{inputenc}
\usepackage[spanish,provide=*]{babel}
\usepackage{amsmath}
\usepackage{amsfonts}
\usepackage{amssymb}
\usepackage{geometry}
\usepackage{xcolor}
\usepackage{hyperref}

\geometry{left=2.5cm, right=2.5cm, top=2.5cm, bottom=2.5cm}
\hypersetup{
    colorlinks=true,
    linkcolor=blue,
    filecolor=magenta,      
    urlcolor=cyan,
}

\title{\textbf{\Large Tema 1.1.1: Razones y Proporciones}}
\author{Tutor Gemini}
\date{\today}

\begin{document}

\maketitle

\section{El Concepto Fundamental: La Raz\'{o}n}

Vamos a empezar con la idea central. Una \textbf{raz\'{o}n} es, en esencia, una forma de \textbf{comparar dos cantidades}. Nos permite entender la relaci\'{o}n que existe entre ellas mediante una divisi\'{o}n.

Imagina que en un sal\'{o}n de clases hay 10 hombres y 15 mujeres. Si queremos expresar la relaci\'{o}n entre ambos, usamos una raz\'{o}n. La raz\'{o}n de hombres a mujeres es de 10 a 15. Podemos escribirla de dos maneras:
\begin{itemize}
    \item Como una fracci\'{o}n: $ \frac{10}{15} $
    \item Usando dos puntos: $ 10 : 15 $
\end{itemize}

Ahora, casi siempre es \'{u}til simplificar esta raz\'{o}n para entenderla mejor. Tanto 10 como 15 son divisibles entre 5. Si simplificamos, obtenemos:
$$ \frac{10 \div 5}{15 \div 5} = \frac{2}{3} $$
Esto nos da una visi\'{o}n mucho m\'{a}s clara: \textbf{por cada 2 hombres en el sal\'{o}n, hay 3 mujeres}. La relaci\'{o}n subyacente es de 2 a 3.

\section{La Herramienta Clave: La Proporci\'{o}n}

Una \textbf{proporci\'{o}n} es simplemente la afirmaci\'{o}n de que dos razones son iguales. Es establecer un equilibrio entre dos comparaciones.

Siguiendo el ejemplo anterior, la raz\'{o}n $ \frac{10}{15} $ es igual a la raz\'{o}n $ \frac{2}{3} $. Por lo tanto, la siguiente igualdad es una proporci\'{o}n:
$$ \frac{10}{15} = \frac{2}{3} $$

\subsection*{La Propiedad de Productos Cruzados}

Aqu\'{i} est\'{a} la parte m\'{a}s importante y la que usar\'{a}s constantemente para resolver problemas. En toda proporci\'{o}n, el producto de los t\'{e}rminos en diagonal (cruzados) es igual.

$$ \text{Si } \frac{a}{b} = \frac{c}{d} \text{, entonces se cumple que } a \cdot d = b \cdot c $$

Esta propiedad es fundamental porque nos permite encontrar un valor desconocido (una inc\'{o}gnita, $x$) en una proporci\'{o}n.

\textbf{Ejemplo pr\'{a}ctico:} Si sabes que 5 boletos para un concierto cuestan \$1200, ¿cu\'{a}nto costar\'{a}n 8 boletos?

\begin{enumerate}
    \item \textbf{Planteamos la proporci\'{o}n.} La relaci\'{o}n (o raz\'{o}n) entre boletos y costo debe ser constante.
    $$ \frac{5 \text{ boletos}}{1200 \text{ pesos}} = \frac{8 \text{ boletos}}{x \text{ pesos}} $$
    (Nuestra inc\'{o}gnita 'x' es el costo de los 8 boletos).

    \item \textbf{Aplicamos productos cruzados.}
    $$ 5 \cdot x = 1200 \cdot 8 $$
    $$ 5x = 9600 $$

    \item \textbf{Despejamos la inc\'{o}gnita.}
    $$ x = \frac{9600}{5} = 1920 $$
\end{enumerate}
\textbf{Resultado:} 8 boletos costar\'{a}n \$1920.

\section{Un Caso Especial: Los Porcentajes}

Los porcentajes son un tipo de raz\'{o}n que usamos todos los d\'{i}as. Un \textbf{porcentaje} es una raz\'{o}n en la que el denominador (el total) siempre es 100. La expresi\'{o}n "por ciento" (\%) significa literalmente "de cada 100".

Por ejemplo, un 40\% de descuento significa que por cada \$100 del precio original, te descuentan \$40. La raz\'{o}n es $ \frac{40}{100} $.

\textbf{Aplic\'{a}ndolo a tus resultados:} En el diagn\'{o}stico de matem\'{a}ticas, tuviste 15 aciertos de un total de 39. Para expresar esto como un porcentaje, buscamos una raz\'{o}n equivalente con denominador 100.

$$ \frac{15 \text{ aciertos}}{39 \text{ totales}} = \frac{x}{100} $$

Usando productos cruzados:
$$ 15 \cdot 100 = 39 \cdot x $$
$$ 1500 = 39x $$
$$ x = \frac{1500}{39} \approx 38.46 $$
El resultado es \textbf{38.46\%}. Este n\'{u}mero es nuestro punto de partida; nos indica exactamente d\'{o}nde debemos concentrar nuestro esfuerzo para mejorar.

\newpage

\section{Ejercicios de Pr\'{a}ctica}

Te recomiendo que intentes resolver estos problemas por tu cuenta. Es la mejor forma de comprobar si el m\'{e}todo ha quedado claro. Las soluciones est\'{a}n detalladas m\'{a}s abajo.

\begin{enumerate}
    \item Un autom\'{o}vil gasta 9 litros de gasolina para recorrer 120 km. Si en el tanque quedan 6 litros, ¿cu\'{a}ntos kil\'{o}metros m\'{a}s podr\'{a} recorrer?

    \item Para construir un muro, 8 obreros tardan 6 d\'{i}as. ¿Cu\'{a}ntos obreros se necesitar\'{i}an para construir el mismo muro en tan solo 4 d\'{i}as? (\textit{Nota}: Este es un caso de \textbf{proporcionalidad inversa}. Si una cantidad sube, la otra baja. La l\'{o}gica es un poco distinta).

    \item Una tienda ofrece un descuento del 25\% en un par de zapatos que originalmente costaban \$800. ¿Cu\'{a}nto pagar\'{a}s por los zapatos despu\'{e}s del descuento?

    \item En un mapa, la escala indica que 2 cm representan 5 km de la realidad. Si la distancia entre dos ciudades en el mapa es de 15 cm, ¿cu\'{a}l es la distancia real entre ellas?
\end{enumerate}

\section*{Soluciones a los Ejercicios}

\begin{enumerate}
    \item \textbf{Soluci\'{o}n:}
    Establecemos la proporci\'{o}n: $ \frac{9 \text{ litros}}{120 \text{ km}} = \frac{6 \text{ litros}}{x \text{ km}} $.
    Aplicamos productos cruzados: $ 9 \cdot x = 120 \cdot 6 $, de donde $9x = 720 $.
    Resolvemos para x: $ x = \frac{720}{9} = 80 $.
    \textbf{Respuesta:} Podr\'{a} recorrer 80 km.

    \item \textbf{Soluci\'{o}n (Proporcionalidad Inversa):}
    En este caso, la relaci\'{o}n no es una divisi\'{o}n, sino una multiplicaci\'{o}n constante: (obreros) $ \times $ (d\'{i}as) = K.
    Calculamos el total de "trabajo": $ 8 \text{ obreros} \times 6 \text{ d\'{i}as} = 48 \text{ obreros-d\'{i}a} $.
    Ahora, con el nuevo dato: $ x \text{ obreros} \times 4 \text{ d\'{i}as} = 48 $.
    Resolvemos para x: $ x = \frac{48}{4} = 12 $.
    \textbf{Respuesta:} Se necesitar\'{i}an 12 obreros.

    \item \textbf{Soluci\'{o}n:}
    Primero, calculamos el monto del descuento. El 25\% de \$800 es $ 0.25 \times 800 = 200 $.
    El precio final es el precio original menos el descuento: \$800 - \$200 = \$600 $.
    \textbf{Respuesta:} Pagar\'{a}s \$600.

    \item \textbf{Soluci\'{o}n:}
    Establecemos la proporci\'{o}n: $ \frac{2 \text{ cm}}{5 \text{ km}} = \frac{15 \text{ cm}}{x \text{ km}} $.
    Aplicamos productos cruzados: $ 2 \cdot x = 5 \cdot 15 $, de donde $ 2x = 75 $.
    Resolvemos para x: $ x = \frac{75}{2} = 37.5 $.
    \textbf{Respuesta:} La distancia real es de 37.5 km.
\end{enumerate}

\end{document}